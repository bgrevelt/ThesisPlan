\cite{moszynski2013novel}\\
Moszynski et al. describe a method to compress MBES data (which includes water column data) using Huffman coding. The proposed algorithm is based on static Huffman encoding with two adaptations to improve performance:
\begin{itemize}
\item The algorithm does not creates a Huffman tree once for each message type instead of for every message. The authors assume that similar probabilities among different messages of the same type. 
\item The algorithm encodes sample data in its true resolution whereas the original format stores the data in a higher resolution than the actual resolution which causes a waste of storage.
\end{itemize}

The average compression ration is approximately 1:3 and outperforms general purpose compression methods ZIP, 7-ZIP and RAR in both compression rate and compression time. 
There are some exceptions, specifically when settings of the MBES are changed during the survey. In that case, the Hoffman tree is no longer optimal (since it was created for the first packet) and compression rate drops.

Interesting points
\begin{itemize}
\item The data sets used to test the algorithm all originate from a Reson 7125. It is unclear how the algorithm performs for other data sources.
\item All datasets are gathered at a location with "relatively flat and homogenous bathymetry". It would be interesting to see how the algorithm would perform with more information in the water column (such as fish, wrecks or gas plumes). 
\item The authors mention that other compression techniques offer greater efficiency: "Although  Huffman  coding  is  optimal  for  a  symbol-by-
symbol  coding  with  a  known  input  probability  distribution,  
its efficiency has since been surpassed by several compression 
algorithms  developed  later.  (...) These coding 
techniques often have better compression capability although 
they are characterized by greater computing complexity and 
larger amount of memory that needs to be allocated during data 
compression and decompression.". It would be interesting to compare this algorithm to one of the more efficient techniques to compare the compression rate and processing time.
\item The results show that the algorithm performs poorly when MBES settings are changed during the survey. This leads to two questions:
\begin{enumerate}
\item How common is it to change MBES settings during a survey?
\item How would compression ratio and time be affected if the Huffman tree was created for every message?
\end{enumerate} 
\end{itemize}