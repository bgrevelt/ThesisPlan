\chapter{Timeline}
\chapterdecscription{An overview of what activity will take place when, and what milestones/deadlines the project has.}

% \documentclass[tikz]{standalone}
% \usepackage{pgfgantt}
% \title{Gantt Charts with the pgfgantt Package}
% \begin{document}

% %
% % A fairly complicated example from section 2.9 of the package
% % documentation. This reproduces an example from Wikipedia:
% % http://en.wikipedia.org/wiki/Gantt_chart
% %
% \definecolor{barblue}{RGB}{153,204,254}
% \definecolor{groupblue}{RGB}{51,102,254}
% \definecolor{linkred}{RGB}{165,0,33}
% \renewcommand\sfdefault{phv}
% \renewcommand\mddefault{mc}
% \renewcommand\bfdefault{bc}
% \setganttlinklabel{s-s}{START-TO-START}
% \setganttlinklabel{f-s}{FINISH-TO-START}
% \setganttlinklabel{f-f}{FINISH-TO-FINISH}
% \sffamily
% \begin{ganttchart}[
%     canvas/.append style={fill=none, draw=black!5, line width=.75pt},
%     hgrid style/.style={draw=black!5, line width=.75pt},
%     vgrid={*1{draw=black!5, line width=.75pt}},
%     today=7,
%     today rule/.style={
%       draw=black!64,
%       dash pattern=on 3.5pt off 4.5pt,
%       line width=1.5pt
%     },
%     today label font=\small\bfseries,
%     title/.style={draw=none, fill=none},
%     title label font=\bfseries\footnotesize,
%     title label node/.append style={below=7pt},
%     include title in canvas=false,
%     bar label font=\mdseries\small\color{black!70},
%     bar label node/.append style={left=2cm},
%     bar/.append style={draw=none, fill=black!63},
%     bar incomplete/.append style={fill=barblue},
%     bar progress label font=\mdseries\footnotesize\color{black!70},
%     group incomplete/.append style={fill=groupblue},
%     group left shift=0,
%     group right shift=0,
%     group height=.5,
%     group peaks tip position=0,
%     group label node/.append style={left=.6cm},
%     group progress label font=\bfseries\small,
%     link/.style={-latex, line width=1.5pt, linkred},
%     link label font=\scriptsize\bfseries,
%     link label node/.append style={below left=-2pt and 0pt}
%   ]{1}{13}
%   \gantttitle[
%     title label node/.append style={below left=7pt and -3pt}
%   ]{WEEKS:\quad1}{1}
%   \gantttitlelist{2,...,13}{1} \\
%   \ganttgroup[progress=57]{WBS 1 Summary Element 1}{1}{10} \\
%   \ganttbar[
%     progress=75,
%     name=WBS1A
%   ]{\textbf{WBS 1.1} Activity A}{1}{8} \\
%   \ganttbar[
%     progress=67,
%     name=WBS1B
%   ]{\textbf{WBS 1.2} Activity B}{1}{3} \\
%   \ganttbar[
%     progress=50,
%     name=WBS1C
%   ]{\textbf{WBS 1.3} Activity C}{4}{10} \\
%   \ganttbar[
%     progress=0,
%     name=WBS1D
%   ]{\textbf{WBS 1.4} Activity D}{4}{10} \\[grid]
%   \ganttgroup[progress=0]{WBS 2 Summary Element 2}{4}{10} \\
%   \ganttbar[progress=0]{\textbf{WBS 2.1} Activity E}{4}{5} \\
%   \ganttbar[progress=0]{\textbf{WBS 2.2} Activity F}{6}{8} \\
%   \ganttbar[progress=0]{\textbf{WBS 2.3} Activity G}{9}{10}
%   \ganttlink[link type=s-s]{WBS1A}{WBS1B}
%   \ganttlink[link type=f-s]{WBS1B}{WBS1C}
%   \ganttlink[
%     link type=f-f,
%     link label node/.append style=left
%   ]{WBS1C}{WBS1D}
% \end{ganttchart}

%
% A simpler example from the package documentation:
%

\begin{ganttchart}[y unit chart=8mm]{1}{12}
  \gantttitle{Week of thesis}{12} \\
  \gantttitle{1}{1} \gantttitle{2}{1}  \gantttitle{3}{1}  \gantttitle{4}{1}  \gantttitle{5}{1}  \gantttitle{6}{1}  \gantttitle{7}{1}  \gantttitle{8}{1}  \gantttitle{9}{1}  \gantttitle{10}{1}  \gantttitle{11}{1}  \gantttitle{12}{1}  
  \\
    \ganttbar{Metrics selection}{1}{5} \\
    \ganttbar{Data selection}{1}{5} \\
    \ganttbar{Benchmark algorithm}{6}{9} \\
    \ganttbar{Benchmark procedure}{10}{11}\\
    \ganttbar{Benchmark validation}{12}{12}
    
    \ganttlink{elem0}{elem2}
    \ganttlink{elem1}{elem2}
    \ganttlink{elem2}{elem4}
    \ganttlink{elem2}{elem3}
\end{ganttchart}

\section{Phases}
\subsection{Metrics selection}
In this phase the metrics that the benchmark should calculate will be selected. This will be based which metrics are reported in (WCD) compression algorithm publications, and the metrics reported in benchmarks in other (similar) domains. 

If possible, interviews will be held with authors of WCD compression algorithms to get information on what they feel are important metrics for these algorithms.

\subsection{Data selection}
In this phase we will select the data that is used in the benchmark. This may be real-life data, generated data or a combination of both. This phase will include a literature study on the applications of WCD in the hydrographic domain and a literature survey on data selection in other benchmarks. 

Like the metrics selection, an interview with experts in the field will be conducted if possible to get information on what they feel are choke points in the domain. Based on these choke points, additional data may be selected for inclusion.

\subsection{Benchmark algorithm}
In this phase, the algorithm of the bench mark will be designed and implemented. This includes definition of interfaces for the algorithms, calculation of metrics and definition of the benchmark result format.

\subsection{Benchmark procedure}
In this phase the procedures for a user of the benchmark are defined. This includes
\begin{itemize}
\item How to prepare a WCD compression algorithm for benchmarking.
\item How to benchmark an algorithm.
\item How to publish the results of the benchmark.
\end{itemize}




% \temporary{

% Phases:
% \begin{enumerate}
% \item Contact authors\\Contact the authors to see if we can get them to share algorithm(s) and data sets they used in their experiment.
% \item Input set definition\\Define which data sets are needed for the experiment. This is (likely) based on applications of water column data and MBES vendor/type occurrence.
% \item Acquire input data sets\\Find data sources for the set definition made in the previous step.
% \item Write data set preprocessing\\Water column data is often interspersed with other hydrographic data in the original file format. The water column packets themselves may also have been broken up into a number of sub-packets that need to be combined into a single packet before compression.\\This step can also be used to get statistics from the data such as the number of pings in the data set, the minimum/maximum depth in the data set, etc
% \item Replicate studies\\Attempt to replicate the studies either based on the paper or on the information we have received from the authors. The results of this step will show if we get the same results as the authors. The results of these step are important for interpretation of the results of the next step: If the results from the original study cannot be replicated, the results of the comparative study may not be appropriate (for instance when an error was introduced when trying to replicate the compression algorithm).
% \item Apply complete data set\\The comparative study. A broad data set is applied to all three compression algorithms. 
% \begin{itemize}
% \item Adapt the original algorithm to accept all encodings used in the data set.
% \item Build a framework that can apply any of the compression algorithms and returns
% \begin{itemize}
% \item The compression factor achieved by the algorithm for the complete data set
% \item The compression factor achieved by the algorithm per ping (on average and per individual ping)
% \item The CPU time used by the algorithm for the complete data set
% \item The CPU time used by the algorithm per ping (on average and per individual ping)
% \item If the data after decompression is equal to that before compression
% \end{itemize}
% \end{itemize}
% \item Analyses\\Analyse the results, see if correlation exists between compression rate and/or time on one side and encoding, device and/or type of information in the data on the other hand.
% \item Improvement\\Based on the results of the analyses we may be able to improve one or more of the algorithms.
% \end{enumerate}

% \begin{ganttchart}[y unit chart=8mm]{1}{12}
%   \gantttitle{Timeline}{12} \\
%   \ganttgroup{Preparation}{1}{2} \\  
%     \ganttbar{Contact authors}{1}{1} \\
%     \ganttbar{Input set definition}{1}{1} \\
%     \ganttbar{Acquire input data sets}{1}{1} \\
%     \ganttlinkedbar{Write Data set preprocessing}{2}{2}\\
%     \ganttlink{elem2}{elem4}
%   \ganttgroup{Replication}{3}{4} \\    
%     \ganttbar{Replicate studies}{3}{3} \\
%     \ganttlink{elem4}{elem6}
%     \ganttlinkedbar{Apply complete data set}{4}{4} \\
%   \ganttgroup{Analyses \& improvement}{5}{6} \\  
%     \ganttbar{Analyses}{5}{5} \\
%     \ganttlink{elem7}{elem9}
%     \ganttlinkedbar{Improvement}{6}{6}\\
   
   
% % \ganttlink{elem3}{elem4} % Data set preprocessing to Replication group
% % \ganttlink{elem5}{elem9} % replication to extension 1 
% % \ganttlink{elem6}{elem10} % replication to extension 2
% % \ganttlink{elem7}{elem11} % replication to extension 3
% % \ganttlink{elem9}{elem12} % extension 1 to analyses
% % \ganttlink{elem10}{elem12} % extension 2 to analyses
% % \ganttlink{elem11}{elem12} % extension 3 to analyses
 
% %  \gantttitlelist{1,...,12}{1} \\
% %  \ganttgroup{Group 1}{1}{7} \\
% % \ganttbar{Task 1}{1}{2} \\
% %  \ganttlinkedbar{Task 2}{3}{7} \ganttnewline
% %  \ganttmilestone{Milestone}{7} \ganttnewline
% %  \ganttbar{Final Task}{8}{12}
% %  \ganttlink{elem2}{elem3}
% %  \ganttlink{elem1}{elem2}
% \end{ganttchart}
%}
% \end{document}